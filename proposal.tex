\documentclass[a4paper,11pt]{article}

\title{CS422 Project Proposal: Searching for Markovian Malware C\&C on Twitter}
\author{Maxime Augier}

\begin{document}

\maketitle

C\&C, short for Command-and-Control, is the infrastructure used by a Botnet (a network of hosts compromised by malicious software) to receive
commands and otherwise communicate with the owners of the malicious software. It is often the most critical part of a malware system; disabling it or taking it over can make a whole botnet self-destruct or become ineffective.

Consequently, C\&C channels have been the subject of a technological race between malware writers and defenders, the former resorting to more and more sophisticated techniques to avoid detection and disabling.

Twitter is a good candidate for such a C\&C channel: An administrator cannot block \url{twitter.com} without possibly alienating users. However, because existing botnets use encrypted, robot-looking messages, they can easily be distinguished from legitimate tweets, and defenders could try to detect and block that.

The aim of the project is to determine how easy and realistic would it be to build an innocuous-looking C\&C channel, by using simple language models such as Markov chains, to masquerade control messages as human utterances; and try to detect occurrences of such behaviour in the wild, by looking for messages that fit an artificial language model particularily well.

First step would be to use an existing corpus of Twitter messages to build a simple Markov model. Then, we must estimate how much entropy one can fit in a single tweet with such a model, with the desired probability on commands. Lastly, we try to detect messages issued by such a model, for instance by looking for messages that have high output probability under a Markov model of depth $n$, but low probability under a model of depth $n+1$.

the Twitter Stream as provided by \url{archive.org} is roughly 50GB per month. Using as many months as we can afford will give us a more accurate model. I don't know yet if this can be achieved with existing software out-of-the-box, but I look forward to learning that in class. This looks suitable for MapReduce-style computations.

I've been considering working on that alone for a while, but if that can be turned into a team project with one another person, it might give me the motivation needed to go forward!

\end{document}